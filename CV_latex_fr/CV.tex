% CV.tex
% !TEX TS-program = pdflatex
% !TEX encoding = UTF-8 Unicode
\documentclass[10pt, oneside, a4paper, titlepage]{article}

\usepackage[most]{tcolorbox}
\usepackage{tikz}
\usepackage{graphicx}
\usepackage[]{geometry}
\usepackage{booktabs}
\usepackage{setspace}
\geometry{
    a4paper,
    left=0.1cm,
    right=0.1cm,
    top=0.2cm,
    bottom=0.1cm
    }
    \renewcommand{\baselinestretch}{0.95}
    \definecolor{titleBack}{RGB}{0,128,255} % Blue
    
    \title{Tobias SAVARY}
    \date{}
    
    \begin{document}

    % Create the rectangle at the top of the CV 
    \tcbset{colframe=gray!95!black, colback=titleBack, arc = 3mm}
    
    \begin{tcolorbox}
        % Use to fill in the top
        \begin{minipage}{0.3\linewidth}
            %This is the place for the picture of your face
            % \hspace*{-0.3cm}\includegraphics[scale = 0.5]{img/TOBIAS SAVARY Photo identité Oct2020.jpg}
            \begin{tikzpicture} 

                \begin{scope}
                    \clip [rounded corners=.5cm] (0,0) rectangle coordinate (centerpoint) (3.5,5cm); 
                    \node [inner sep=0pt] at (centerpoint) {\includegraphics[width=4.0cm]{img/TOBIAS SAVARY Photo identité Oct2020.jpg}}; 
                \end{scope}
            \end{tikzpicture}
        \end{minipage}%
        \hspace{1cm}%%
        \begin{minipage}{0.6\linewidth}
            \begin{center}
                \Huge{\textcolor{white}{Tobias SAVARY}} \\
                \vspace*{0.5cm}
                \Large{\textcolor{white}{\emph{Etudiant en Génie Informatique \\Université de Technologie de Compiègne (UTC)}}}
            \end{center}
        \end{minipage}%
    \end{tcolorbox}

    \tcbset{colframe=white, colback=white, arc = 2mm}
    \begin{tcolorbox}
        \vspace*{0.2cm}
        \hspace*{0.7mm}
        \begin{minipage}[t]{7cm}
            \begin{spacing}{0.95}
            \vspace*{-0.5cm}
            \begin{tcolorbox}[grow to left by = 0.6cm, colback = gray!25, colframe = white]
                % \section*{Profile}
                %     Here you can see my profile and I will complete it next time.
                
                \section*{\\Coordonnées}
                \hspace*{0.4cm}
                10 Bis Rue de Chateau Rouge

                \hspace*{0.4cm}
                \vspace*{0.2cm}
                60730 Cauvigny\\
                \hspace*{0.4cm}
                Tel: 07 65 21 77 91\\
                \hspace*{0.4cm}
                Email: savarytobias@hotmail.com\\

                \section*{Programmation}
                \begin{itemize}
                    \item Python
                    \item C
                    \item Ocaml
                    \item ARM
                    \item HTML
                    \item CSS
                    \item \LaTeX
                \end{itemize}
                \vspace*{1mm}
                \section*{Compétences}

                \begin{itemize}
                    \item Utilisation aisée des logiciels bureautiques (Word / Excel / Paint/ \\PowerPoint / \LaTeX \ldots);

                    \item Utilisation aisée d’applications \\mobiles diverses;

                    \item Maîtrise des langages de \\programmation: Python, C, Ocaml, HTML, CSS, ARM. \\
                \end{itemize}



                \section*{Activités et intérêts}

                \begin{itemize}
                    \item Participation aux manifestations de communication proposées par l’école (Journées Portes Ouvertes, Journée du lycéen, présentation de la formation auprès des élèves de Terminal dans les lycées de l’agglomération);

                    \item Cyclisme: participation à divers évènements et rencontres sportives;

                    \item Natation, Badminton, 
                    
                    Tennis de Table. \\
                \end{itemize}

            \end{tcolorbox}
        \end{spacing}
        \end{minipage}
        \hspace*{0.4mm}
        \begin{minipage}[t]{12cm}
            \vspace*{-0.5cm}
            \begin{tcolorbox}[grow to right by = 0.6cm, colback = gray!25, colframe = white]
                \section*{Projets informatiques réalisés}
                \begin{itemize}
                    \item Réalisation de jeux de plateau en Python (scrabble, Futoshiki);
                    \item Gestion d’ensembles mathématiques en Ocaml;
                    \item Traitement d’arbres phylogénétiques, décodage de message crypté, déplacement d’un robot, simplification de contours d’images en C;
                    \item Codage de divers petites pages web avec HTML et CSS.

                \end{itemize}
                
                \section*{Expériences professionnelles}
                \begin{itemize}
                    \item \textbf{2022}: Stage d'excellence de 2 mois au sein du laboratoire INRIA 
                    de Grenoble. - Réalisation d'une 
                    interface graphique pour développer un jeu d'apprentissage
                    du débogage;
                    \item \textbf{2021}: Stages au sein des services départementaux de l’Oise:
                    \begin{itemize}
                        \item Agent administratif - Comité des Œuvres Sociales [COS]
                        \item Agent administratif polyvalent - Maison Départementale des \\Personnes Handicapées de l’Oise [MDPH]
                        \item Agent d’accueil - Agence Départementale d’Information sur le Logement de l’Oise [ADIL60]

                    \end{itemize}
                    \item \textbf{2017}: Stage d’observation en informatique au sein de la Direction Numérique du Conseil Départemental de l'Oise;
                    
                    \item \textbf{2017-2022}: Garde d’enfants et cours particuliers.
                \end{itemize}

                \section*{Diplômes et formations }
                \begin{itemize}
                    \item \textbf{2022}: Etudiant en cycle ingénieur Génie Informatique à l'UTC;             
                    \item \textbf{2020-2022}: Formation préparatoire intégrée au réseau d’écoles d’ingénieurs Polytech, adossée à la licence du parcours "Mathématique et Informatique" de l’Université Grenoble Alpes [UGA], dont les enseignements transverses (Anglais, gestion de projet, exploration professionnel\ldots). \emph{Classement: 120ème sur 1 564 étudiants};
                    \item \textbf{2020}: Permis de conduire (Permis B);

                    \item \textbf{2020}: Baccalauréat (Série S – Section Européenne Anglais - Mention bien);                  
                    \item \textbf{2018}: Certification Cambridge English;
                    \item \textbf{2017}: Diplôme National du Brevet (Mention très bien).
                \end{itemize}

                    \section*{Langues}
                    Anglais (B2) et Allemand (A2)
                    \begin{itemize}
                        \item Séjours linguistiques : en Angleterre (famille d’accueil) et \\en Allemagne (échanges scolaires).
                    \end{itemize}
                    \vspace*{0.1mm}
            \end{tcolorbox}
        \end{minipage}
    \end{tcolorbox}
\end{document}